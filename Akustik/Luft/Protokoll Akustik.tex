\documentclass[]{article}
\usepackage{amsmath}
\usepackage{amsfonts}
\usepackage{amssymb}
\usepackage[utf8]{inputenc}
\usepackage{graphicx}
\usepackage{geometry}
\usepackage{color}
\usepackage{siunitx} %\SI{44.9}{\celsius}
\usepackage[german]{babel}
\usepackage{fancyref}
\usepackage{circuitikz} %für Stromkreise
\long\def \/*#1*/{} %Kommentare mit \/* ---- */
\geometry{
	a4paper,
	left=25mm,
	right=25mm,
	top=25mm,
	bottom=25mm,
}
\pagenumbering{arabic}

\title{Messung der Schallgeschwindigkeit in Luft}
\date{02.01.18}
\author{Mate Farkas, Patrick Schillings}
\begin{document}
	
	
	\tableofcontents
	
	\noindent\makebox[\linewidth]{\rule{\textwidth}{0.4pt}}
	
	\section{Versuchsziele}
	
	Das Ziel des Versuchs ist es, die Schallgeschwindigkeit in Luft zu messen. Dazu sollen drei verschiedene Methoden verwendet werden, einmal die Auftragung einer Laufstrecke innerhalb einer Laufzeit, einmal die Messung von Resonanzfrequenzen innerhalb eines Rohres bekannter Länge und einmal durch die Messung der Wellenlänge bei bekannter Frequenz einer stehenden Welle.
	
	\section{Grundlagen} %TODO: Als eigenes?
	
	\begin{equation}
		v_{Schall}=\lambda*f=\frac{\Delta s}{\Delta t}
	\end{equation}
	
	\begin{equation}
		L=\frac{\lambda}{2} 
	\end{equation}	
	
	%Billd stehende Welle
	
	\section{Versuche}
	
	\subsection{Messung von Laufzeit und Laufstrecke eines Signals}
	\subsubsection{Aufbau und Durchführung}
	\subsubsection{Auswertung}
	
	\subsection{Messung von Resonanzfrequenzen einer stehenden Welle}
	\subsubsection{Aufbau und Durchführung}
	\subsubsection{Auswertung}
	
	\subsection{Messung der Wellenlänge einer stehenden Welle}
	\subsubsection{Aufbau und Durchführung}
	\subsubsection{Auswertung}
	
	\section{Ergebnisse}
	
	
\end{document}