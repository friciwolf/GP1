\documentclass[]{article}
\usepackage{amsmath}
\usepackage{amsfonts}
\usepackage{amssymb}
\usepackage[utf8]{inputenc}
\usepackage{graphicx}
\usepackage{geometry}
\usepackage{color}
\usepackage{siunitx} %\SI{44.9}{\celsius}
\usepackage[german]{babel}
\usepackage{fancyref}
\usepackage{circuitikz} %für Stromkreise
\long\def \/*#1*/{} %Kommentare mit \/* ---- */
\geometry{
	a4paper,
	left=25mm,
	right=25mm,
	top=25mm,
	bottom=25mm,
}
\pagenumbering{arabic}

\title{Messung der Schallgeschwindigkeit in Luft}
\date{02.01.18}
\author{Mate Farkas, Patrick Schillings}
\begin{document}
	
	
	\tableofcontents
	
	\noindent\makebox[\linewidth]{\rule{\textwidth}{0.4pt}}
	
	\section{Versuchsziele}
	
	Das Ziel des Versuchs ist es, die Schallgeschwindigkeit in Luft zu messen. Dazu sollen drei verschiedene Methoden verwendet werden, einmal die Auftragung einer Laufstrecke innerhalb einer Laufzeit, einmal die Messung von Resonanzfrequenzen innerhalb eines Rohres bekannter Länge und einmal durch die Messung der Wellenlänge bei bekannter Frequenz einer stehenden Welle.
	
	\section{Grundlagen} %TODO: Als eigenes?
	
	Die Wellenlänge $\lambda$ und die Frequenz $f$ beschreiben eine Schallwelle mit der Geschwindigkeit
	
	\begin{equation}
		v_{Schall}=\lambda*f=\frac{\Delta s}{\Delta t}.
		\label{e1}
	\end{equation}
	
	Für stehende Wellen auf der Länge $L$ mit zwei festen Enden gilt
	
	\begin{equation}
		L=n*\frac{\lambda}{2}
		\label{e2}
	\end{equation}
	oder umgeformt
	\begin{equation}
	v_{Schall}=\frac{2L}{n}*f
	\label{e3} 
	\end{equation}	
	mit der Ordnung n.
	
	%Bild stehende Welle
	
	\section{Versuche}
	
	\subsection{Kalibration des Wegaufnehmers}
	\subsubsection{Aufbau und Durchführung}
	
	Um die stehende Welle im Rohr in den späteren Teilen zu untersuchen, wurde der Wegaufnehmer gegenüber eines Messbandes Klasse II kalibriert. Ziel dieses Teilversuches ist damit, einen funktionalen Zusammenhang zwischen den am Wegaufnehmer (der wie ein Schiebewiderstand charakterisieren lässt) gemessenen Widerstand und die Längenverschiebung L. Dazu wurde der {\color{red} {Schiebekopf}} auf die 0.5 cm - Marke des Messbandes gelegt und der Widerstand mithilfe eines Teststroms gemessen. Dies wurde in 5-cm Schritten wiederholt, wovon die Rohdaten sich in der folgenden Tabelle zusammenfassen lassen:
	
	\begin{center}
		
		\begin{tabular}{|c|c|c|c|c|c|c|}
			\hline 
			Abstand s in cm & 0.5 & 5.5 & 10.5 & 15.5 & 20.5 & 25.5 \\ 
			\hline 
			Widerstand R in $\Omega$ & 1.685 & 1.375, & 1.06 & 0.755, & 0.44 & 0.125 \\ 
			\hline 
		\end{tabular} 
	\end{center}
	
	\subsubsection{Auswertung}
	
	Die Messpunkte Widerstand R gegen Position L des Mikrophons kann man in Abb.\ref{Kalib_Reg} erkennen. Mit diesen wurde eine lineare Regression $L=k*R+L_0$ durchgeführt, die ebenfalls dargestellt ist. Als Fehler gehen hierbei die Ablese- beziehungsweise Digitalisierungslimitierungen ein: $L_{err}=0.1cm/2$ und $R_{err}=0.005k\Omega/\sqrt{12}$ ein. Der konstante systematische Fehler des Maßbandes spielt keine Rolle, da später nur die Steigung des Graphen, sowie Längendifferenzen wichtig werden.      
     
    Das Ergebnis der Regression beträgt $(16.040\pm0.042)cm/k\Omega*R+(2.957\pm0.044)cm$ mit einem $X^2/f \approx 2.704/4 \approx 0.676$. Der Residuenplot dazu ist in Abb.\ref{Kalib_Res} dargestellt. Man erkennt, dass alle Fehler $\sqrt{k^2*R_{err}^2+L_{err}^2}$ etwa im Rahmen einer Standardabweichung um die Funktion gestreut sind.\\   
                   
   	\begin{figure}
    	\begin{center}
    		\includegraphics[scale=0.9]{Images/Kalibrierung_Regression.pdf}
    		\caption{Regressionsgerade der Kalibrierung}             
    		\label{Kalib_Reg}               
    	\end{center}            
    \end{figure} 
    
	\begin{figure}
	\begin{center}
		\includegraphics[scale=0.9]{Images/Kalibrierung_Residuum.pdf}
		\caption{Residuenplot der Kalibrationsgeraden}
		\label{Kalib_Res}
	\end{center}
	\end{figure}
	\subsection{Messung von Laufzeit und Laufstrecke eines Signals}
	In der ersten Versuchsreihe wurden die Laufzeiten und -wege wie in Abb.\ref{Va_Reg} aufgenommen. Hier hilft der zweite Teil von Formel \ref{e1}, um die Geschwindigkeit des Schalls in Luft mit einer linearen Regression $s(t)=v_{Schall}*t+s_0$ zu bestimmen. 
	
	\begin{figure}
	\begin{center}
		\includegraphics[scale=0.9]{Images/VA_Regression.pdf}
		\caption{Regression zur Ermittlung der Schallgeschwindigkeit}
		\label{Va_Reg}
	\end{center}
	\end{figure}	
	Das Residuum kann man in Abb.\ref{Va_Res} sehen. Die Werte liegen gestreut um die Gerade, so wie es auch sein sollte, allerdings 
	\begin{figure}
	\begin{center}
		\includegraphics[scale=0.9]{Images/VA_Residuum.pdf}
		\caption{Residuum der Geradenanpassung}
		\label{Va_Res}
	\end{center}
	\end{figure}	


	
	\subsection{Messung von Resonanzfrequenzen einer stehenden Welle}
	\subsubsection{Aufbau und Durchführung}
	\subsubsection{Auswertung}
	
	\subsection{Messung der Wellenlänge einer stehenden Welle}
	\subsubsection{Aufbau und Durchführung}
	\subsubsection{Auswertung}
	
	Die Messergebnisse kann man in Abb.\ref{Vc_Roh} betrachten.

	\begin{figure}
	\begin{center}
		\includegraphics[scale=0.9]{Images/VC_Roh.png}
		\caption{Schallamplitude bei bestimmten Positionen}
		\label{Vc_Roh}
	\end{center}
	\end{figure}
	

	\section{Ergebnisse}
	\subsection{Temperaturauswertung}
	%\section{Anhang} ?
	
	\begin{figure}
	\begin{center}
		\includegraphics[scale=0.9]{Images/Rauschmessung_vor_den_Messungen.pdf}
		\caption{Rauschmessung der Temperatur vor den Messungen}
		\label{Temp_vorn}
	\end{center}
	\end{figure}	

	\begin{figure}
	\begin{center}
		\includegraphics[scale=0.9]{Images/Rauschmessung_in_der_Mitte.pdf}
		\caption{Rauschmessung der Temperatur zwischen Experimenten}
		\label{Temp_mitte}
	\end{center}
	\end{figure}

	\begin{figure}
	\begin{center}
		\includegraphics[scale=0.9]{Images/Rauschmessung_nach_den_Messungen.pdf}
		\caption{Rauschmessung der Temperatur nach den Messungen}
		\label{Temp_hinten}
	\end{center}
	\end{figure}

	\begin{figure}
	\begin{center}
		\includegraphics[scale=0.9]{Images/Temperatur_vor_den_Messungen.pdf}
		\caption{Histogramm der Temperatur vor den Messungen}
		\label{Temp_vorn_hist}
	\end{center}
	\end{figure}

	\begin{figure}
	\begin{center}
		\includegraphics[scale=0.9]{Images/Temperatur_in_der_Mitte.pdf}
		\caption{Histogramm der Temperatur zwischen den Messungen}
		\label{Temp_mitte_hist}
	\end{center}
	\end{figure}

	\begin{figure}
	\begin{center}
		\includegraphics[scale=0.9]{Images/Temperatur_nach_den_Messungen.pdf}
		\caption{Histogramm der Temperatur nach den Messungen}
		\label{Temp_hinten_hist}
	\end{center}
	\end{figure}

	\begin{figure}
	\begin{center}
		\includegraphics[scale=0.9]{Images/Temperatur_Verlauf.pdf}
		\caption{Anpassung einer Exponentialfunktion an den Temperaturverlauf}
		\label{Temp_Fit}
	\end{center}
	\end{figure}

	\begin{figure}
	\begin{center}
		\includegraphics[scale=0.9]{Images/Temperatur_Residuum.pdf}
		\caption{Residuenplot der linearen Regression an die logarrithmierte Temperatur}
		\label{Temp_Res}
	\end{center}
	\end{figure}




	
\end{document}
